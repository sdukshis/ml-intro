\documentclass[14pt]{extarticle}

\usepackage[utf8]{inputenc}
\usepackage[russian]{babel}
\usepackage{cmap}
\usepackage{hyperref}

\title{Python для анализа данных}

\author{Павел Филонов \\ \href{mailto:filonovpv@gmail.com}{filonovpv@gmail.com}}

\begin{document}
\maketitle

\section*{Введение}
Для решения задач машинного обучения необходим инструментарий, который должен включать в себя следующие основные элементы:
    \begin{itemize}
        \item основные матричные операции;
        \item загрузка и сохранение данных;
        \item библиотеки стандартных алгоритмов из машинного обучения.
    \end{itemize}

Существует множество различных инструментов, которые удовлетворяют таким требованиями. Из них стоит отметить 
    \begin{itemize}
        \item Язык программирования R;
        \item MATLAB;
        \item SPSS.
    \begin{itemize}

В данном пособии рассмотрено как применять специализированные библиотеки языка {\it Python}\cite[python] для задач анализа данных. По ходе изложения будут рассмотрены элементы таких библиотек как
    \begin{itemize}
        \item NumPy \cite{numpy}
        \item Pandas \cite{pandas}
        \item SciPy \cite{scipy}
        \item Matplotlib \cite{matplotlib}
        \item SciKit-learn \cite{sklearn}
    \end{itemize}
и работа в среде {\it Jupyter Notebook}\cite{jupyter}.

\section{Основные инструменты для работы с данными}
\subsection{Numpy}
\subsection{Pandas}

%\section{Метрические методы классификации}

%\section{Линейная регрессия}
\clearpage
\tableofcontents
\end{document}