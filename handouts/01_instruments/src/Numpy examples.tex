% Default to the notebook output style

% Inherit from the specified cell style.




    
\documentclass{article}

    
    

    \usepackage{graphicx} % Used to insert images
    \usepackage{adjustbox} % Used to constrain images to a maximum size 
    \usepackage{color} % Allow colors to be defined
    \usepackage{enumerate} % Needed for markdown enumerations to work
    \usepackage{geometry} % Used to adjust the document margins
    \usepackage{amsmath} % Equations
    \usepackage{amssymb} % Equations
    \usepackage{eurosym} % defines \euro
    \usepackage[mathletters]{ucs} % Extended unicode (utf-8) support
    \usepackage[utf8x]{inputenc} % Allow utf-8 characters in the tex document
    \usepackage{fancyvrb} % verbatim replacement that allows latex
    \usepackage{grffile} % extends the file name processing of package graphics 
                         % to support a larger range 
    % The hyperref package gives us a pdf with properly built
    % internal navigation ('pdf bookmarks' for the table of contents,
    % internal cross-reference links, web links for URLs, etc.)
    \usepackage{hyperref}
    \usepackage{longtable} % longtable support required by pandoc >1.10
    \usepackage{booktabs}  % table support for pandoc > 1.12.2
    \usepackage{ulem} % ulem is needed to support strikethroughs (\sout)
     % load all other packages
% For cyrillic symbols
\usepackage[english, russian]{babel}


    
    \definecolor{orange}{cmyk}{0,0.4,0.8,0.2}
    \definecolor{darkorange}{rgb}{.71,0.21,0.01}
    \definecolor{darkgreen}{rgb}{.12,.54,.11}
    \definecolor{myteal}{rgb}{.26, .44, .56}
    \definecolor{gray}{gray}{0.45}
    \definecolor{lightgray}{gray}{.95}
    \definecolor{mediumgray}{gray}{.8}
    \definecolor{inputbackground}{rgb}{.95, .95, .85}
    \definecolor{outputbackground}{rgb}{.95, .95, .95}
    \definecolor{traceback}{rgb}{1, .95, .95}
    % ansi colors
    \definecolor{red}{rgb}{.6,0,0}
    \definecolor{green}{rgb}{0,.65,0}
    \definecolor{brown}{rgb}{0.6,0.6,0}
    \definecolor{blue}{rgb}{0,.145,.698}
    \definecolor{purple}{rgb}{.698,.145,.698}
    \definecolor{cyan}{rgb}{0,.698,.698}
    \definecolor{lightgray}{gray}{0.5}
    
    % bright ansi colors
    \definecolor{darkgray}{gray}{0.25}
    \definecolor{lightred}{rgb}{1.0,0.39,0.28}
    \definecolor{lightgreen}{rgb}{0.48,0.99,0.0}
    \definecolor{lightblue}{rgb}{0.53,0.81,0.92}
    \definecolor{lightpurple}{rgb}{0.87,0.63,0.87}
    \definecolor{lightcyan}{rgb}{0.5,1.0,0.83}
    
    % commands and environments needed by pandoc snippets
    % extracted from the output of `pandoc -s`
    \providecommand{\tightlist}{%
      \setlength{\itemsep}{0pt}\setlength{\parskip}{0pt}}
    \DefineVerbatimEnvironment{Highlighting}{Verbatim}{commandchars=\\\{\}}
    % Add ',fontsize=\small' for more characters per line
    \newenvironment{Shaded}{}{}
    \newcommand{\KeywordTok}[1]{\textcolor[rgb]{0.00,0.44,0.13}{\textbf{{#1}}}}
    \newcommand{\DataTypeTok}[1]{\textcolor[rgb]{0.56,0.13,0.00}{{#1}}}
    \newcommand{\DecValTok}[1]{\textcolor[rgb]{0.25,0.63,0.44}{{#1}}}
    \newcommand{\BaseNTok}[1]{\textcolor[rgb]{0.25,0.63,0.44}{{#1}}}
    \newcommand{\FloatTok}[1]{\textcolor[rgb]{0.25,0.63,0.44}{{#1}}}
    \newcommand{\CharTok}[1]{\textcolor[rgb]{0.25,0.44,0.63}{{#1}}}
    \newcommand{\StringTok}[1]{\textcolor[rgb]{0.25,0.44,0.63}{{#1}}}
    \newcommand{\CommentTok}[1]{\textcolor[rgb]{0.38,0.63,0.69}{\textit{{#1}}}}
    \newcommand{\OtherTok}[1]{\textcolor[rgb]{0.00,0.44,0.13}{{#1}}}
    \newcommand{\AlertTok}[1]{\textcolor[rgb]{1.00,0.00,0.00}{\textbf{{#1}}}}
    \newcommand{\FunctionTok}[1]{\textcolor[rgb]{0.02,0.16,0.49}{{#1}}}
    \newcommand{\RegionMarkerTok}[1]{{#1}}
    \newcommand{\ErrorTok}[1]{\textcolor[rgb]{1.00,0.00,0.00}{\textbf{{#1}}}}
    \newcommand{\NormalTok}[1]{{#1}}
    
    % Additional commands for more recent versions of Pandoc
    \newcommand{\ConstantTok}[1]{\textcolor[rgb]{0.53,0.00,0.00}{{#1}}}
    \newcommand{\SpecialCharTok}[1]{\textcolor[rgb]{0.25,0.44,0.63}{{#1}}}
    \newcommand{\VerbatimStringTok}[1]{\textcolor[rgb]{0.25,0.44,0.63}{{#1}}}
    \newcommand{\SpecialStringTok}[1]{\textcolor[rgb]{0.73,0.40,0.53}{{#1}}}
    \newcommand{\ImportTok}[1]{{#1}}
    \newcommand{\DocumentationTok}[1]{\textcolor[rgb]{0.73,0.13,0.13}{\textit{{#1}}}}
    \newcommand{\AnnotationTok}[1]{\textcolor[rgb]{0.38,0.63,0.69}{\textbf{\textit{{#1}}}}}
    \newcommand{\CommentVarTok}[1]{\textcolor[rgb]{0.38,0.63,0.69}{\textbf{\textit{{#1}}}}}
    \newcommand{\VariableTok}[1]{\textcolor[rgb]{0.10,0.09,0.49}{{#1}}}
    \newcommand{\ControlFlowTok}[1]{\textcolor[rgb]{0.00,0.44,0.13}{\textbf{{#1}}}}
    \newcommand{\OperatorTok}[1]{\textcolor[rgb]{0.40,0.40,0.40}{{#1}}}
    \newcommand{\BuiltInTok}[1]{{#1}}
    \newcommand{\ExtensionTok}[1]{{#1}}
    \newcommand{\PreprocessorTok}[1]{\textcolor[rgb]{0.74,0.48,0.00}{{#1}}}
    \newcommand{\AttributeTok}[1]{\textcolor[rgb]{0.49,0.56,0.16}{{#1}}}
    \newcommand{\InformationTok}[1]{\textcolor[rgb]{0.38,0.63,0.69}{\textbf{\textit{{#1}}}}}
    \newcommand{\WarningTok}[1]{\textcolor[rgb]{0.38,0.63,0.69}{\textbf{\textit{{#1}}}}}
    
    
    % Define a nice break command that doesn't care if a line doesn't already
    % exist.
    \def\br{\hspace*{\fill} \\* }
    % Math Jax compatability definitions
    \def\gt{>}
    \def\lt{<}
    % Document parameters
    \title{Numpy examples}
    
    
    

    
    % Prevent overflowing lines due to hard-to-break entities
    \sloppy 
    % Setup hyperref package
    \hypersetup{
      breaklinks=true,  % so long urls are correctly broken across lines
      colorlinks=true,
      urlcolor=blue,
      linkcolor=darkorange,
      citecolor=darkgreen,
      }
    % Slightly bigger margins than the latex defaults
    
    \geometry{verbose,tmargin=1in,bmargin=1in,lmargin=1in,rmargin=1in}
    
    

    \begin{document}
    
    
    \maketitle
    
    

    
    Импортируем пакет NumPy с псевдонимом np

    Установим число символов после запятой при печати

    Массивы numpy можно создавать как из плоских списков

    
    
    \begin{verbatim}
array([2, 5, 0])
    \end{verbatim}

    

    так и из вложеных

    
    
    \begin{verbatim}
array([[2, 5, 0],
       [3, 5, 1],
       [3, 7, 8]])
    \end{verbatim}

    

    Для создания нулевых векторов и матриц можно использовать функцию zeros

    
    
    \begin{verbatim}
array([ 0.,  0.,  0.])
    \end{verbatim}

    

    
    
    \begin{verbatim}
array([[ 0.,  0.,  0.],
       [ 0.,  0.,  0.],
       [ 0.,  0.,  0.]])
    \end{verbatim}

    

    Для матриц состоящих из одних единиц есть аналогочная функция ones.

    
    
    \begin{verbatim}
array([ 1.,  1.,  1.])
    \end{verbatim}

    

    
    
    \begin{verbatim}
array([[ 1.,  1.,  1.],
       [ 1.,  1.,  1.],
       [ 1.,  1.,  1.]])
    \end{verbatim}

    

    С помощью функции eye можно создать единичную матрицу нужной размерности

    
    
    \begin{verbatim}
array([[ 1.,  0.,  0.],
       [ 0.,  1.,  0.],
       [ 0.,  0.,  1.]])
    \end{verbatim}

    

    Для генерации диапазонов чисел удобно использовать функцию arange

    
    
    \begin{verbatim}
array([0, 1, 2, 3, 4, 5, 6, 7, 8, 9])
    \end{verbatim}

    

    Её параметры позволяют указать начало и конец полуинтервала, а также шаг
и тип данных.

    
    
    \begin{verbatim}
array([ 2.,  4.,  6.,  8.])
    \end{verbatim}

    

    Если нужно получить равномерную сетку, то удобней воспользоваться
функцией linspace

    
    
    \begin{verbatim}
array([ 1.   ,  1.333,  1.667,  2.   ,  2.333,  2.667,  3.   ,  3.333,
        3.667,  4.   ])
    \end{verbatim}

    

    Часто для примера бывает нужно быстро создать матрицу, заполненную
случайными числами.

    
    
    \begin{verbatim}
array([[ 0.406,  0.249,  0.851],
       [ 0.226,  0.324,  0.718],
       [ 0.464,  0.984,  0.2  ]])
    \end{verbatim}

    

    Одним из мощнейших инструментов работы с многомерными массивами являются
срезы (slices)

    
    
    \begin{verbatim}
array([[ 0.746,  0.793,  0.335],
       [ 0.115,  0.633,  0.895],
       [ 0.562,  0.281,  0.259]])
    \end{verbatim}

    

    
    
    \begin{verbatim}
array([ 0.746,  0.115,  0.562])
    \end{verbatim}

    

    
    
    \begin{verbatim}
array([ 0.115,  0.633,  0.895])
    \end{verbatim}

    

    С помощью срезов можно организовать присваивания по строкам или столбцам

    
    
    \begin{verbatim}
array([[ 0.793,  0.793,  0.335],
       [ 0.633,  0.633,  0.895],
       [ 0.281,  0.281,  0.259]])
    \end{verbatim}

    

    Для транспонирования матриц в пакете numpy имеется функция transpose

    
    
    \begin{verbatim}
array([[ 0.793,  0.633,  0.281],
       [ 0.793,  0.633,  0.281],
       [ 0.335,  0.895,  0.259]])
    \end{verbatim}

    

    Для сохранения матриц в файл и загрузки из файла в память можно
использовать соответвующие функции

    
    
    \begin{verbatim}
array([[ 0.369,  0.736],
       [ 0.262,  0.795],
       [ 0.696,  0.58 ],
       [ 0.597,  0.023],
       [ 0.704,  0.091],
       [ 0.189,  0.598],
       [ 0.296,  0.703],
       [ 0.651,  0.951],
       [ 0.316,  0.776],
       [ 0.515,  0.221]])
    \end{verbatim}

    

    Особое внимание следует обратить на то, что при использовании обычных
арифметических операторов будут произведены поэлементные операции

    
    
    \begin{verbatim}
array([ 1.235,  1.195,  0.867])
    \end{verbatim}

    

    
    
    \begin{verbatim}
array([-0.137, -0.487, -0.769])
    \end{verbatim}

    

    
    
    \begin{verbatim}
array([ 0.376,  0.298,  0.04 ])
    \end{verbatim}

    

    
    
    \begin{verbatim}
array([ 0.8  ,  0.421,  0.06 ])
    \end{verbatim}

    

    
    
    \begin{verbatim}
array([[ 0.24 ,  0.029,  0.019],
       [ 0.545,  0.238,  0.035],
       [ 0.354,  0.258,  0.008]])
    \end{verbatim}

    

    Чтобы выполнить матричное умножение следует использовать функцию dot

    
    
    \begin{verbatim}
array([ 0.289,  0.817,  0.621])
    \end{verbatim}

    

    Для решения СЛАУ, заданного в матричном виде используется функция solve

    
    
    \begin{verbatim}
array([ 0.,  0., -0.])
    \end{verbatim}

    

    Для обращения матриц используется функция inv

    
    
    \begin{verbatim}
array([[ 1.,  0.,  0.],
       [ 0.,  1., -0.],
       [-0.,  0.,  1.]])
    \end{verbatim}

    

    Рассмотрим простой пример использования обращения матриц на основе
решения задачи линейной регресии методом наименьших квадратов.

Вначале сгенерируем исходные данные с помощью функций linspace и внесем
шум с помощью np.random.randn

    Визуализируем результат генерации в виде точек.

    \begin{center}
    \adjustimage{max size={0.9\linewidth}{0.9\paperheight}}{Numpy examples_files/Numpy examples_50_0.png}
    \end{center}
    { \hspace*{\fill} \\}
    
    В соответствии с методом наименьших квадратов вычислим коэфициенты
регресии как \[
    b = (X^T X)^{-1}X^T y
\] здесь \(X\) - расширенная матрица

\[
    X = \begin{pmatrix}
        x_1 & 1 \\
        x_2 & 1 \\
        \vdots \\
        x_n & 1
    \end{pmatrix}
\]

    Вычислим точки линейной регресси для визуализации

    и нарисуем все вместе

    \begin{center}
    \adjustimage{max size={0.9\linewidth}{0.9\paperheight}}{Numpy examples_files/Numpy examples_56_0.png}
    \end{center}
    { \hspace*{\fill} \\}
    

    % Add a bibliography block to the postdoc
    
    
    
    \end{document}
